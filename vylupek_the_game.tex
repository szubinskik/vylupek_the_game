% % % % % % % % % % % % % % % %
% Fiszki z teorii statystyki  %
% by: Marcin Sidorowicz       %
% % % % % % % % % % % % % % % %

\documentclass[avery5371, grid, frame]{flashcards}

\usepackage[utf8]{inputenc}
\usepackage[T1]{fontenc}
\usepackage{amsmath}
\usepackage{amssymb}
\usepackage{enumitem}

\geometry{headheight=12pt}
\usepackage{fancyhdr}
\pagestyle{fancy}
\fancyhf{}
\renewcommand{\headrulewidth}{0pt}

\cardfrontstyle[]{headings}
\cardfrontfoot{Teoria statystyki}


\begin{document}

\begin{flashcard}[Definicja]{Rozkład gamma \\ (parametry, gęstość, MGF)}
    Parametry: $\alpha > 0, \beta > 0$
    $$ f(x) =
    \begin{cases}
        \frac{1}{\Gamma(\alpha)\beta^\alpha} x^{\alpha-1} e^{-x/\beta}, \quad x \in (0, \infty) \\
        0, \text{w p.p.}
    \end{cases} $$
    $$M_X(t) = \left ( \frac{1}{1-\beta t}\right )^\alpha, t < \frac{1}{\beta}$$
\end{flashcard}

\begin{flashcard}[Definicja]{Rozkład chi kwadrat \\ (parametry, gęstość, MGF)}
    Parametry: $r > 0$
    $$ f(x) =
    \begin{cases}
        \frac{1}{\Gamma(r/2)2^{r/2}} x^{r/2 - 1} e^{-x/2}, x \in (0, \infty) \\
        0, \text{w p.p.}
    \end{cases} $$
    $$M_X(t) = (1-2t)^{-r/2}, t < \frac{1}{2}$$
\end{flashcard}

\begin{flashcard}[Definicja]{Rozkład beta \\ (parametry, gęstość)}
    Parametry: $\alpha > 0, \beta > 0$
    $$ f(x) =
    \begin{cases}
        \frac{\Gamma(\alpha + \beta)}{\Gamma(\alpha) \Gamma(\beta)} x^{\alpha - 1} (1-x)^{\beta - 1}, x \in (0, 1) \\
        0, \text{w p.p.}
    \end{cases} $$
\end{flashcard}

\begin{flashcard}[Twierdzenie]{Zależności między rozkładami beta, gamma, chi kwadrat}
\begin{small}
    \begin{itemize}
        \item $\Gamma (r/2, 2) \,{\buildrel d \over =}\, \chi^2 (r)$
        \item $\Gamma(1, \frac{1}{\lambda}) \,{\buildrel d \over =}\, \mathcal{E}(\lambda) $
        \item $\sum_{i = 1}^{n} \Gamma(\alpha_i, \beta) \,{\buildrel d \over =}\, \Gamma(\sum_{i=1}^{n} \alpha_i, \beta) $
        \item $\sum_{i=1}^{n} \chi^2(r_i) \,{\buildrel d \over =}\, \chi^2 (\sum_{i=1}^{n} r_i)$
    \end{itemize}
\end{small}
\end{flashcard}

\begin{flashcard}[Definicja]{Rozkład normalny \\ (parametry, gęstość, MGF)}
    Parametry: $\mu \in \mathbb{R}$, $\sigma^2 > 0$
    $$ f(x) = \frac{1}{\sigma \sqrt{2\pi}} e^{\frac{-(x-\mu)^2}{2\sigma^2}} $$
    $$ M_X(t) = e^{\mu t + \frac{1}{2}\sigma^2 t^2} $$
\end{flashcard}

\begin{flashcard}[Definicja]{Wielowymiarowy rozkład normalny \\ (parametry, gęstość, MGF)}
    Parametry: $\boldsymbol{\mu} \in \mathbb{R}^n$, $\boldsymbol{\Sigma} \in M_{n \times n}(\mathbb{R})$ symetryczna, dodatnio określona
    $$ f(\mathbf{x}) = \frac{1}{(2\pi)^{n/2} |\boldsymbol{\Sigma}|^{1/2}} \text{exp} \left \{ -\frac{1}{2} (\mathbf{x} - \boldsymbol{\mu})^T \boldsymbol{\Sigma} (\mathbf{x} - \boldsymbol{\mu}) \right \}$$
    $$M_x(\mathbf{t}) = \text{exp}\{ \mathbf{t}^T \boldsymbol{\mu} + \frac{1}{2} \mathbf{t}^T \boldsymbol{\Sigma} \boldsymbol{t}\} $$
\end{flashcard}

\begin{flashcard}[Twierdzenie]{Przekształcenie liniowe wielowymiarowego rozkładu normalnego}
    Niech \textbf{X} ma rozkład $\mathcal{N}_n (\boldsymbol{\mu}, \boldsymbol{\Sigma})$. Niech $\mathbf{A} \in M_{m \times n}(\mathbb{R})$, $\mathbf{b} \in \mathbb{R}^m$. \\
    Wtedy $\mathbf{AX} + \mathbf{b}$ ma rozkład $\mathcal{N}_m(\mathbf{A}\boldsymbol{\mu} + \mathbf{b}, \mathbf{A}\boldsymbol{\Sigma}\mathbf{A}^T)$.
\end{flashcard}

\begin{flashcard}[Twierdzenie]{Rozkłady brzegowe w wielowymiarowym rozkładzie normalnym}
    Niech \textbf{X} ma rozkład $\mathcal{N}_n (\boldsymbol{\mu}, \boldsymbol{\Sigma})$, gdzie \\
    $ \boldsymbol{\mu} =
    \left[ \begin{array}{c}
    \boldsymbol{\mu}_1 \\
    \boldsymbol{\mu}_2 \end{array} \right] $,
    $ \boldsymbol{\Sigma} =
    \left[ \begin{array}{cc}
    \boldsymbol{\Sigma}_{11} & \boldsymbol{\Sigma}_{12}\\
    \boldsymbol{\Sigma}_{21} & \boldsymbol{\Sigma}_{22} \end{array} \right] $. \newline

    Wtedy $\mathbf{X_1}$ ma rozkład $\mathcal{N}_m(\boldsymbol{\mu_1}, \boldsymbol{\Sigma}_{11})$. \\ Ponadto $\mathbf{X_1}$, $\mathbf{X_2}$ są niezależne wtw., gdy $\boldsymbol{\Sigma}_{12} = \mathbf{0}$.
\end{flashcard}

\begin{flashcard}[Definicja]{Rozkład \textit{t}-Studenta \\ (parametry, i otrzymywanie z innych rozkładów)}
    Parametr: $r > 0$ \\
    Niech $W \sim \mathcal{N}(0, 1)$, $V \sim \chi^2(r)$. Wtedy $$ T = \frac{W}{\sqrt{V/r}} $$ ma rozkład Studenta z $r$ stopniami swobody.
\end{flashcard}

\begin{flashcard}[Twierdzenie]{Twierdzenie Studenta}
\begin{small}
\begin{flushleft}
    Niech $X_1, X_2, \dots, X_n$ będą i.i.d. zmiennymi z rozkładu $\mathcal{N}(\mu, \sigma^2)$. Oznaczmy $\overline{X} = \frac{1}{n} \sum_{i=1}^{n} X_i$, $S^2 = \frac{1}{n-1} \sum_{i=1}^{n} (X_i - \overline{X})^2$. Wtedy:
    \begin{itemize}
        \item $\overline{X}$ ma rozkład $\mathcal{N}(\mu, \frac{\sigma^2}{n})$.
        \item $\overline{X}$ i $S^2$ są niezależne.
        \item $(n-1)S^2/\sigma^2$ ma rozkład $\chi^2(n-1)$.
        \item Zmienna $ T = \frac{\overline{X} - \mu}{S / \sqrt{n}} $ ma rozkład Studenta z $n-1$ stopniami swobody.
    \end{itemize}
\end{flushleft}
\end{small}
\end{flashcard}

\begin{flashcard}[Definicja]{Rozkład \textit{F}-Snedecora \\ (parametry i otrzymywanie z innych rozkładów)}
    Parametry: $r_1$, $r_2 > 0$ \\
    Niech $U \sim \chi^2(r_1)$, $V \sim \chi^2(r_2)$. Wtedy zmienna $$ W = \frac{U/r_1}{V/r_2} $$ ma rozkład Snedecora z $r_1$ stopniami swobody w liczniku i $r_2$ w mianowniku.
\end{flashcard}

\begin{flashcard}[Definicja]{Zbieżność wg prawdopodobieństwa}
    Mówimy, że ciąg zmiennych losowych $\{ X_n \}$ zbiega wg prawdodpobieństwa do $X$, jeżeli dla każdego $\epsilon > 0$ $$\lim_{n \rightarrow \infty} P(|X_n - X| \geq \epsilon) = 0.$$
\end{flashcard}

\begin{flashcard}[Twierdzenie]{Słabe prawo wielkich liczb}
    Niech $\{X_n\}$ będzie ciągiem i.i.d. zmiennych losowych o średniej $\mu$ i wariancji $\sigma^2 < \infty$. Wtedy $$ \frac{1}{n} \sum_{i=1}^{n} X_i \xrightarrow{P} \mu.$$
\end{flashcard}

\begin{flashcard}[Definicja]{Estymator zgodny}
    Niech zmienna $X$ ma gęstość $f(x, \theta), \theta \in \Omega$. Niech $X_1, X_2, \dots, X_n$ będzie próbką z rozkładu $X$ i niech $T_n$ oznacza statystykę. $T_n$ nazywamy zgodnym estymatorem $\theta$, jeżeli $$ T_n \xrightarrow{P} \theta . $$
\end{flashcard}

\begin{flashcard}[Definicja]{Zbieżność wg rozkładu}
     Mówimy, że ciąg zmiennych losowych $\{ X_n \}$ zbiega wg rozkładu do $X$, jeżeli dla każdego punktu ciągłości $x$ dystrybuanty $F_X$ mamy $$ \lim_{n \rightarrow \infty} F_{X_n}(x) = F_X(x).$$
\end{flashcard}

\begin{flashcard}[Definicja]{Ograniczenie w prawdopodobieństwie}
    Mówimy, że ciąg zmiennych losowych $\{ X_n \}$ jest ograniczony w prawdopodobieństwie, jeżeli dla każdego $\epsilon > 0$ istnieją $B_\epsilon > 0$ i indeks $N_\epsilon$ takie, że $$ n \geq N_\epsilon \Rightarrow P(|X_n| \leq B_\epsilon) \geq 1-\epsilon$$.
\end{flashcard}

\begin{flashcard}[Twierdzenie]{Twierdzenia o $\Delta$-metodzie}
    \begin{itemize}
        \item Niech $\{Y_n\}$ będzie ciągiem zmiennych losowych ograniczonych w prawdopodobieństwie. Załóżmy, że $X_n = o_p(Y_n)$. Wtedy $X_n \xrightarrow{P} 0$.
        \item Niech $\{X_n\}$ będzie ciągiem zmiennych losowych takich, że $ \sqrt{n}(X_n - \theta) \xrightarrow{D} \mathcal{N}(0, \sigma^2).$ Załóżmy, że $g(x)$ jest różniczkowalna w $\theta$ i $g'(\theta) \neq 0.$ Wtedy $\sqrt{n}(g(X_n) - g(\theta)) \xrightarrow{D} \mathcal{N}(0, \sigma^2(g'(\theta))^2)$.
    \end{itemize}
\end{flashcard}

\begin{flashcard}[Twierdzenie]{Zbieżność MGF \\ a zbieżność wg rozkładu}
    Niech $\{X_n\}$ będzie ciągiem zmiennych losowych takich, że $M_{X_n}(t)$ istnieją dla $t \in (-h, h)$. Niech $X$ będzie zmienną losową taką, że $M_X(t)$ istnieje dla $t \in [-h_1, h_1] \subseteq [-h, h]$. Jeżeli dla $t \in [-h_1, h_1]$ mamy $\lim_{n \rightarrow \infty} M_{X_n}(t) = M_X(t)$, to $X_n \xrightarrow{D} X$.
\end{flashcard}

\begin{flashcard}[Twierdzenie]{Centralne twierdzenie graniczne}
    Niech $\{X_n\}$ będzie ciągiem i.i.d. zmiennych losowych z rozkładu o średniej $\mu$ i wariancji $0 < \sigma^2 < \infty$. Wtedy $$ \frac{\sum_{i=1}^{n} X_i - n\mu}{\sigma \sqrt{n}} = \frac{\sqrt{n}(\overline{X}_n - \mu)}{\sigma} \xrightarrow{D} \mathcal{N}(0, 1).$$
\end{flashcard}

\begin{flashcard}[Twierdzenie]{Rozkład łączny i brzegowy statystyk porządkowych}
    $$ g(y_1, y_2, \dots, y_n) =
    \begin{cases}
        n! \prod_{i=1}^{n}f(y_i), y_i \text{ rosnąco} \\
        0, \text{w p.p.}
    \end{cases} $$
    \begin{align*}
        g_k(y_k) = \int_a^{y_2} \dots \int_a^{y_k} \int_{y_k}^b \dots \int_{y_{n-1}}^b n! \prod f(y_i) \\ dy_n \dots dy_{k+1} dy_{k-1} \dots dy_1.
    \end{align*}
\end{flashcard}

\begin{flashcard}[Definicja]{Estymator największej wiarogodności (MLE)}
    Mówimy, że $\hat{\theta} = \hat{\theta}(\mathbf{X}$ jest estymatorem największej wiarygodności $\theta$, jeżeli $$ \hat{\theta} = Argmax L(\theta; \mathbf{X}),$$ gdzie $ L(\theta;\textbf{X}) = \prod_{i=1}^n f(x_i; \theta), \theta \in \Omega$ (funkcja wiarygodności).
\end{flashcard}

\begin{flashcard}[Twierdzenie]{MLE funkcji parametru}
    Niech $X_1, X_2, \dots, X_n$ będzie próbą z rozkładu \\ o gęstości $f(x; \theta)$, $\theta \in \Omega$. Załóżmy że $\hat{\theta}$ jest MLE parametru $\theta$. Wtedy $g(\hat{\theta})$ jest MLE parametru $\eta = g(\theta)$.
\end{flashcard}

\begin{flashcard}[Twierdzenie]{Warunki na zgodność MLE}
    Załóżmy, że rodzina $f(x;\theta)$ jest różna dla różnych $\theta$, ma wspólny nośnik i faktyczny parametr $\theta_0$ jest we wnętrzu $\Omega$. Wtedy $ \frac{\partial}{\partial \theta} L(\theta) = 0 $ ma rozwiązanie $\theta_n$ będące zgodnym estymatorem $\theta$.
\end{flashcard}

\begin{flashcard}[Definicja]{Warunki regularności rodziny $f(x; \theta)$}
\begin{footnotesize}
\setlist[enumerate, 1]{start=0}
\begin{enumerate}
    \item Funkcje gęstości są rozróżnialne, tzn. $\theta \neq \theta' \Rightarrow f(x_i; \theta) \neq f(x_i, \theta')$.
    \item Funkcje gęstości mają ten sam nośnik dla wszystkich $\theta$.
    \item $\theta_0$ (faktyczna wartość $\theta$) leży wewnątrz $\Omega$.
    \item $f(x; \theta)$ jest dwukrotnie różniczkowalna po $\theta$.
    \item $\int f(x; \theta) dx$ jest dwukrotnie różniczkowalna pod całką.
    \item $f(x; \theta)$ jest trzykrotnie różniczkowalna po $\theta$. Dla każdej $\theta \in \Omega$ istnieje stała $c$ i funkcja $M(x)$ taka, że $\left| \frac{\partial^3}{\partial \theta^3} \log f(x; \theta) \right| \leq M(x)$, $E_{\theta_0}[M(x)] < \infty$ dla $\theta \in (\theta_0 - c, \theta_0 + c)$.
\end{enumerate}
\end{footnotesize}
\end{flashcard}

\begin{flashcard}[Definicja]{Informacja Fishera (dodatkowe założenia)}
    Założenia: warunki regularności (R0)-(R4) \\
    Dla rozkładu $f(x; \theta)$ określamy
    \begin{multline*}
    I(\theta) = \mathbb{E} \left[  \left( \frac{\partial \log f(X; \theta)}{\partial \theta}\right)^2 \right] = \\ = Var \left( \frac{\partial \log f(X; \theta)}{\partial \theta} \right) = \int_{-\infty}^\infty \frac{\partial^2 \log f(x; \theta)}{\partial \theta^2} f(x; \theta) dx
    \end{multline*}
\end{flashcard}

\begin{flashcard}[Definicja]{Estymator wydajny}
    Mówimy, że $Y$ - zgodny estymator $\theta$ jest wydajny, jeżeli wariancja $Y$ osiąga dolne ograniczenie w nierówności Cram\'era-Rao.
\end{flashcard}

\begin{flashcard}[Twierdzenie]{Nierówność Cram\'era-Rao}
\begin{small}
    Niech zachodzą warunki regularności (R0)-(R4).  Niech $X_1, X_2, \dots, X_n$ będzie próbą z rozkładu o gęstości $f(x; \theta)$, $\theta \in \Omega$. Niech $Y = u(X_1, X_2, \dots, X_n)$ będzie statystyką o średniej $\mathbb{E}[Y] = k(\theta)$. Wtedy $$ Var(Y) \geq \frac{[k'(\theta)]^2}{n I(\theta)}.$$
    W szczególności, dla $Y$ - zgodnego estymatora $\theta$ mamy $$ Var(Y) \geq \frac{1}{n I(\theta)}.$$
\end{small}
\end{flashcard}

\begin{flashcard}[Twierdzenie]{Asymptotyczny rozkład różnicy estymatorów}
    Niech zachodzą warunki regularności (R0)-(R5). Niech $X_1, X_2, \dots, X_n$ będzie próbą z rozkładu $f(x; \theta_0)$ o niezerowej i skończonej informacji Fishera. Jeżeli $g(x)$ jest funkcją ciągłą i różniczkowalną w $\theta_0$, to ciąg zgodnych MLE spełnia
    $$\sqrt{n} (g(\hat{\theta}) - g(\theta_0)) \overset{D}{\rightarrow} \mathcal{N} \left( 0, \frac{g'(\theta_0)^2}{I(\theta_0)} \right). $$
\end{flashcard}

\begin{flashcard}[Definicja]{Wydajność asymptotyczna}
    Niech $\hat{\theta}_{1n}$ będzie zgodnym estymatorem $\theta_0$ takim, że $\sqrt{n}(\hat{\theta}_{1n} - \theta_0) \xrightarrow{D} \mathcal{N}(0, \sigma_{\hat{\theta}_{1n}}^2).$ Wydajnością asymptotyczną nazywamy stosunek $e(\hat{\theta}_{n}) = \frac{1/I(\theta_0)}{\sigma_{\hat{\theta}_{1n}}^2}$. Estymator jest asymptotycznie wydajny, jeżeli ten stosunek wynosi 1. \\
    Asymptotyczna wydajność względna $\hat{\theta}_{1n}$ względem $\hat{\theta}_{2n}$ to stosunek tych wartości.

\end{flashcard}

\begin{flashcard}[Definicja]{Estymator nieobciążony o minimalnej wariancji}
    Estymator nieobciążony o wariancji mniejszej niż dowolny inny estymator nieobciążony.
\end{flashcard}

\begin{flashcard}[Definicja]{Statystyka dostateczna}
    Statystyka $Y_1 = u_1(X_1, X_2, \dots, X_n)$ z próby o rozkładzie (dyskretnym lub ciągłym) $f(x; \theta)$ jest dostateczna wtw., gdy
    $$ \frac{f(x_1; \theta) f(x_2; \theta) \dots f(x_n; \theta)}{f_{Y_1}[u_1(x_1, x_2, \dots, x_n); \theta]} = H(x_1, x_2, \dots, x_n),$$ gdzie funkcja $H$ nie zależy od $\theta$.
\end{flashcard}

\begin{flashcard}[Twierdzenie]{Twierdzenie Neymana}
    Statystyka $Y_1 = u_1(X_1, X_2, \dots, X_n$ z próby o rozkładzie (dyskretnym lub ciągłym) $f(x; \theta)$ jest dostateczna wtw., gdy istnieją dwie nieujemne funkcje $k_1$, $k_2$ takie, że
    \begin{small}
    $$ \prod_{i=1}^n f(x_i; \theta) = k_1[u_1(x_1, x_2, \dots, x_n); \theta] k_2(x_1, x_2, \dots, x_n), $$ gdzie $k_2$ nie zależy od $\theta$.
    \end{small}
\end{flashcard}

\begin{flashcard}[Twierdzenie]{Twierdzenie Rao-Blackwella}
    \begin{small}
    Niech $X_1, X_2, \dots, X_n$ będzie próbą z rozkładu o gęstości (dyskretnej lub ciągłej) $f(x; \theta), \theta \in \Omega$. Niech $Y_1$ będzie statystyką dostateczną dla $\theta$ i niech $Y_2$ będzie nieobciążonym estymatorem $\theta$, który nie jest funkcją samego $Y_1$.\\
    Wtedy statystyka $\varphi(Y_1)$ dana przez funkcję $\varphi(y_1) = \mathbb{E}[Y_2 |Y_1 = y_1]$ jest statystyką dostateczną, nieobciążonym estymatorem $\theta$ i ma wariancję mniejszą niż $Var(Y_2)$.
    \end{small}
\end{flashcard}

\begin{flashcard}[Twierdzenie]{Zależność MLE od statystyki dostatecznej}
    Niech $X_1, X_2, \dots, X_n$ będzie próbą z rozkładu o gęstości (dyskretnej lub ciągłej) $f(x; \theta), \theta \in \Omega$. Jeżeli dla $\theta$ istnieją statystyka dostateczna $Y_1$ oraz jedyny estymator największej wiarogodności $\hat{\theta}$, to $\hat{\theta}$ jest funkcją $Y_1$.
\end{flashcard}

\begin{flashcard}[Definicja]{Zupełna rodzina rozkładów}
    Niech zmienna losowa $Z$ będzie ciągła lub dyskretna o rozkładzie z rodziny $\{h(z; \theta) : \theta \in \Omega\}$. Jeżeli z faktu, że dla każdej $\theta \in \Omega $, $\mathbb{E}[u(Z)] = 0$ można wnioskować, że $u(z) = 0$ wszędzie poza punktami o prawdopodobieństwie zerowym dla każdego $h(z; \theta)$ z tej rodziny, to rodzinę $h(z; \theta)$ nazywamy zupełną rodziną rozkładów.
\end{flashcard}

\begin{flashcard}[Twierdzenie]{Twierdzenie Lehmanna-Scheff\'ego}
    Niech $X_1, X_2, \dots, X_n$ będzie próbą z rozkładu o gęstości (dyskretnej lub ciągłej) $f(x, \theta)$. Niech $Y_1$ będzie statystyką dostateczną dla $\theta$ i niech rodzina $\{ f_{Y_1}(y_1, \theta) : \theta \in \Omega \}$ będzie zupełna. Jeżeli istnieje funkcja od $Y_1$, która jest nieobciążonym estymatorem $\theta$, to jest też jedynym ENMW.
\end{flashcard}

\begin{flashcard}[Definicja]{Dostateczna statystyka zupełna}
    Statystyka $Y_1 = u_1(X_1, X_2, \dots, X_n$ z próby o rozkładzie (dyskretnym lub ciągłym) $f(x; \theta), \theta \in \Omega$ jest dostateczna i zupełna wtw., gdy jest dostateczna i rodzina $\{ f_{Y_1}(y_1, \theta) : \theta \in \Omega \}$ jest zupełna.
\end{flashcard}

\begin{flashcard}[Definicja]{Regularna klasa wykładnicza}
    Mówimy, że rozkład jest w regularnej klasie wykładniczej, jeżeli jest postaci
    $$ f(x; \theta) =
    \begin{cases}
        \text{exp}[p(\theta) K(x) + S(x) + q(\theta)], x \in \mathcal{S} \\
        0, \text{w p.p.,}
    \end{cases} $$
    Jeżeli $\mathcal{S}$ nie zależy od $\theta$, $p(\theta)$ jest ciągłą, nietrywialną funkcją $\theta$, $S$ jest ciągłą funkcją $x$ oraz $K'(x) \not\equiv 0$.
\end{flashcard}

\begin{flashcard}[Twierdzenie]{Własności statystyki z regularnej klasy wykładniczej}
    Niech $X_1, X_2, \dots, X_n$ będzie próbą z rozkładu z regularnej klasy wykładniczej. Wtedy statystyka $Y_1 = \sum_{i=1}^n K(X_i)$ ma własności:
    \begin{enumerate}
        \item $f_{Y_1}(y_1; \theta) = R(y_1) \text{exp}[p(\theta) y_1 + n q(\theta)],$ gdzie nośnik i $R(y_1)$ nie zależą od $\theta$,
        \item $\mathbb{E}[Y_1] = -n \frac{q'(\theta)}{p'(\theta)}$,
        \item $Var(Y_1) = n \frac{1}{p'(\theta)^3} \{ (p''(\theta)q'(\theta) - q''(\theta)p'(\theta) \} $.
    \end{enumerate}
\end{flashcard}

\begin{flashcard}[Twierdzenie]{Zupełna statystyka z regularnej klasy wykładniczej}
    Niech $X_1, X_2, \dots, X_n$ będzie próbą z rozkładu o gęstości $f(x; \theta)$, $\theta \in (\gamma, \delta)$ z regularnej klasy wykładniczej. Wtedy statystyka $Y_1 = \sum_{i=1}^n K(X_i)$ jest dostateczną statystyką zupełną.
\end{flashcard}

\begin{flashcard}[Twierdzenie]{Dostateczny warunek na niezależność statystyki}
    Niech $X_1, X_2, \dots, X_n$ będzie próbą z rozkładu $f(x; \theta), \theta \in \Omega$, gdzie $\Omega$ jest przedziałem. Niech $Y_1$ będzie zupełną statystyką dostateczną dla $\theta$ i niech $Z$ będzie inną statystyką, która nie jest funkcją samego $Y_1$. Jeżeli rozkład $Z$ nie zależy od $\theta$, to $Z$ jest niezależne od $Y_1$.
\end{flashcard}

\begin{flashcard}[Definicja]{Błąd I i II rodzaju}
    Błąd I rodzaju: odrzucenie hipotezy zerowej, gdy jest poprawna. \\
    Błąd II rodzaju: przyjęcie hipotezy zerowej, gdy jest błędna.
\end{flashcard}

\begin{flashcard}[Definicja]{Poziom istotności (i jego interpretacja)}
    Poziom istotności $\alpha$ to liczba z przedziału $[0, 1]$. Interpretujemy to jako akceptowalne ryzyko popełnienia błędu I rodzaju.
\end{flashcard}

\begin{flashcard}[Definicja]{Moc testu statystycznego}
    Mocą testu nazywamy prawdopodobieństwo niepopełnienia błędu II rodzaju.
\end{flashcard}

\begin{flashcard}[Definicja]{Obszar krytyczny}
    Niech $X_1, X_2, \dots, X_n$ będzie testowaną próbą. Obszarem krytycznym $C$ nazywamy taki podzbiór przestrzeni próbek $\mathcal{D}$, że hipotezę zerową odrzucamy wtedy i tylko wtedy, gdy $(X_1, X_2, \dots, X_n) \in C$.
\end{flashcard}

\begin{flashcard}[Definicja]{Wielkość obszaru krytycznego}
    Obszar krytyczny $C$ ma wielkość $\alpha$, jeżeli $$\alpha = \max_{\theta \in \omega_0} P_\theta [(X_1, X_2, \dots, X_n) \in C].$$ Zatem jest to też prawdopodobieństwo popełnienia błędu I rodzaju.
\end{flashcard}

\begin{flashcard}[Definicja]{Najlepszy obszar krytyczny (rozmiaru $\alpha$)}
    Mówimy, że $C$ jest najlepszym obszarem krytycznym rozmiaru $\alpha$ dla prostej hipotezy $H_0 : \theta = \theta'$ przeciw prostej hipotezie $H_1: \theta = \theta''$, jeżeli
    \begin{itemize}
        \item $P_{\theta'}[X \in C] = \alpha$,
        \item $P_{\theta''}[X \in C] \geq P_{\theta''}[X \in A]$ dla każdego innego pozbioru $A$ wielkości $\alpha$ z przestrzeni próbek.
    \end{itemize}
\end{flashcard}

\begin{flashcard}[Twierdzenie]{Twierdzenie Neymana-Pearsona}
\begin{small}
    Niech $X_1, X_2, \dots, X_2$ będzie próbą z rozkładu $f(x; \theta)$. Niech $\Omega = \{\theta', \theta''\}$ i niech $C$ będzie takim podzbiorem przestrzeni próbek, że:\
    \begin{itemize}
        \item $\frac{L(\theta'; x)}{L(\theta''; x)} \leq k$, dla wszystkich $x \in C$,
        \item $\frac{L(\theta'; x)}{L(\theta''; x)} \geq k$, dla wszystkich $x \in C^C$,
        \item $\alpha = P_{H_0}[X \in C]$.
    \end{itemize}
    Wtedy $C$ jest najlepszym obszarem krytycznym wielkości $\alpha$ dla testowania hipotezy $H_0: \theta = \theta'$ przeciw hipotezeie $H_1: \theta = \theta''$.
\end{small}
\end{flashcard}

\begin{flashcard}[Definicja]{Test jednostajnie najmocniejszy}
    Obszar krytyczny $C$ jest obszarem jednostajnie najmocniejszym wielkości $\alpha$ do testowania prostej hipotezy $H_0$ przeciw złożonej hipotezie $H_1$, jeżeli $C$ jest najlepszym obszarem do testowania hipotezy $H_0$ przeciw każdej pojedynczej hipotezie spośród $H_1$. \\
    Test zdefiniowany takim obszarem nazywamy testem jednostajnie najmocniejszym o poziomie istotności $\alpha$.
\end{flashcard}

\end{document}
